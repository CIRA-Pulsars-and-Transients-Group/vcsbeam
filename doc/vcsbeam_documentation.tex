\documentclass{book}

\newcommand{\vcsbeam}{{\sc VCSBeam}}

\title{\vcsbeam{} Documentation}
\author{Dr. Sammy McSweeney}

\usepackage{amsmath}
\usepackage{fullpage}
\usepackage{listings}
\usepackage{graphicx}
\usepackage{hyperref}
\usepackage{natbib}
\usepackage{makeidx}

\setcounter{secnumdepth}{4}
\setcounter{tocdepth}{4}

\hypersetup{
    colorlinks=true,
    linkcolor=blue,
    filecolor=blue,
}

\makeindex

\newcommand{\pd}[2]{\ensuremath{\frac{\partial #1}{\partial #2}}}
\newcommand{\transmat}[4]{\ensuremath{{\bf P}_{(#1,#2)\rightarrow(#3,#4)}}}
\newcommand{\pamat}{\ensuremath{{\bf P}_\text{pa}}}

\begin{document}

\maketitle

\tableofcontents

\chapter{Conventions}

\section{Coordinate systems}

There are three coordinate systems in use throughout \vcsbeam{}:
\begin{enumerate}
    \item Instrumental ($p$,$q$)
    \item Local sky coordinates ($\theta$,$\phi$)
    \item Celestial sky coordinates ($x$,$y$)
\end{enumerate}
They are illustrated in Fig. \ref{fig:coords}.
\begin{figure}[!bh]
    \centering
    \includegraphics[width=\textwidth]{coords.png}
    \caption{Illustration of the three coordinate systems used in this documentation.}
    \label{fig:coords}
\end{figure}

\subsection{Instrumental coordinates}
\index{coordinates!instrumental}

This is a Cartesian coordinate system aligned with local (ground) compass directions.
Positive $p$ points towards local North, and positive $q$ towards local East.
The ``$P$ polarisation'' refers to the physical set of dipoles parallel to the N-S line; the ``$Q$ polarisation'', to the E-W line.

\subsection{Local sky coordinates}
\label{sec:coordslocalsky}
\index{coordinates!local sky}

This is a spherical coordinate system defined with respect to a local observer.
$\theta$ is the \textit{zenith angle}\index{zenith angle}, i.e. a \textit{colatitude}, with zenith itself therefore defined as $\theta = 0$ and the horizon as $\theta = \pi/2$.
$\phi$ is the \textit{azimuth}\index{azimuth}, and we define $\phi = 0$ in the North direction, with positive azimuth moving clockwise as viewed from above (i.e. N$\rightarrow$E$\rightarrow$S$\rightarrow$W$\rightarrow$N).
Moreover, the \textit{elevation}\index{elevation} is denoted by the symbol $\tilde{\theta}$, and is related to the zenith angle by
\begin{equation}
    \tilde{\theta} = \frac{\pi}{2} - \theta.
\end{equation}

\subsection{Celestial sky coordinates}
\index{coordinates!celestial sky}

This is a spherical coordinate system defined with respect to the celestial sphere.
$x$ is the \textit{declination}\index{declination} (Dec) and $y$ is the \textit{right ascension}\index{right ascension} (RA).

\subsection{Comparison of notation in other documents}

See Table \ref{tbl:notations}
\begin{table}[!hb]
    \centering
    \caption{Comparison of notation used elsewhere}
    \label{tbl:notations}
    \begin{tabular}{l|cc|cc|cc}
        & $p$ & $q$ & $\theta$ & $\phi$ & $x$ & $y$ \\
        \hline
        MWA metafits files       & Y & X & - & - & - & - \\
        Sokolowksi et al. (2017) & $y$ & $x$ & $\theta$ & $\phi$ & - & - \\
    \end{tabular}
\end{table}

\subsection{Coordinate transformations}

%The three sets of coordinates described above span two-dimensional subspaces of the ambient three-dimensional space in which the celestial sphere and the telescope reside.
%Consequently, transformations between them generally require $3\times3$ matrices.
%If we define
%\begin{equation}
%    \begin{aligned}
%        \hat{s} \equiv \hat{p} \times \hat{q}, \\
%        \hat{r} \equiv \hat{\theta} \times \hat{\phi}, \\
%        \hat{z} \equiv \hat{x} \times \hat{y},
%    \end{aligned}
%\end{equation}
%then $(p,q,s)$, $(\theta,\phi,r)$, and $(x,y,z)$ are all three-dimensional, right-handed coordinate systems which are related by the following coordinate transformations:
%\begin{equation}
%    \begin{aligned}
%        \theta &= \arctan\left(-\frac{\sqrt{p^2 + q^2}}{s}\right) \\
%        \phi &= \arctan\left(\frac{q}{p}\right) \\
%        r &= -\sqrt{p^2 + q^2 + s^2}
%    \end{aligned}
%\end{equation}

All coordinate transformations can be effected by applying the appropriate Jacobian matrix for the desired transformation.
In this documentation, a boldface ${\bf P}$ will always be used\footnote{``${\bf J}$'' is reserved for (other) Jones matrices.} to denote such transformation matrices.
For general coordinates $(a,b)$ and $(c,d)$,
\begin{equation}
    \transmat{a}{b}{c}{d} =
    \begin{bmatrix}
        \pd{c}{a} & \pd{c}{b} \\[5 pt]
        \pd{d}{a} & \pd{d}{b}
    \end{bmatrix}
\end{equation}
Among these, the only transformation that is explicitly used in \vcsbeam{} is the transformation between local sky coordinates and celestial sky coordinates, which is a single rotation within the sky plane by the parallactic angle.

\subsubsection{Parallactic angle correction}

The parallactic angle correction is a transformation between local sky coordinates and celestial sky coordinates.
The parallactic angle itself, $\chi$, is defined as the position angle of local zenith with respect to the North Celestial Pole as subtended at a given source (see Fig. \ref{fig:skyangles}).
The transformation $(x,y)\rightarrow(\theta,\phi)$ is therefore a counterclockwise rotation\footnote{A counterclockwise rotation of a given vector is equivalent to a clockwise rotation of the coordinate axes.} by $\pi - \chi$.
This is the rotation
\begin{equation}
    \transmat{x}{y}{\theta}{\phi}
        = \begin{bmatrix}
            \cos\left(\pi - \chi\right) & -\sin\left(\pi - \chi\right) \\
            \sin\left(\pi - \chi\right) &  \cos\left(\pi - \chi\right)
        \end{bmatrix}
        = \begin{bmatrix}
            -\cos\chi & -\sin\chi \\
             \sin\chi & -\cos\chi
        \end{bmatrix}.
\end{equation}
\begin{figure}[!th]
    \centering
    \includegraphics[width=0.4\textwidth]{skyangles.png}
    \caption{Definition of the parallactic angle, $\chi$, in the sky plane.}
    \label{fig:skyangles}
\end{figure}
The actual parallactic angle correction that is implemented is for historical reasons the matrix
\begin{equation}
    \pamat \equiv
    \transmat{y}{x}{\theta}{\phi}
        = \begin{bmatrix}
            -\sin\chi & -\cos\chi \\
            -\cos\chi &  \sin\chi
        \end{bmatrix}.
\end{equation}
In \vcsbeam{}, the parallactic angle is calculated (via the function \texttt{palPa} in the \href{https://github.com/Starlink/pal}{Starlink/pal} library) by the spherical triangle identity
\begin{equation}
    \tan \chi = \frac{\cos \lambda \sin H}{\sin\lambda \cos x - \cos \lambda \sin x \cos H},
\end{equation}
where $\lambda$ is the latitude of the observer\footnote{The latitude of the MWA is $\lambda_\text{MWA} = -0.4660608448386394\,$rad, defined in the \texttt{mwalib} library.}, $H$ is the hour angle of the source, and $x$ is the declination.

\chapter{Algorithms}

\section{PFB}

\subsection{Fine channelisation}

\section{Calibration}
\index{calibration}

Calibration is the process by which the temporal variations in the instrumental and/or ionospheric response to an incoming signal are characterised, in order to account for them when processing observational data.
A \textit{calibration solution} is a mathematical operation that can be applied either to measured visibilities or raw voltages to correct for these variations, and recover the visibilities (or raw voltages) that would have been measured under ideal conditions.

For the MWA, calibration solutions are modelled as a tile-dependent, linear transformation of two orthogonal polarisations.
Mathematically, this means that the calibration solution for each tile is a $2\times2$ Jones matrix, which in the $(p,q)$ basis is
\begin{equation}
    \begin{aligned}
        \tilde{\bf v} &= {\bf J}{\bf v}_\text{measured} \\
        \begin{bmatrix} \tilde{v}_p \\ \tilde{v}_q \end{bmatrix}
            &= \begin{bmatrix}
                J_{pp} & J_{pq} \\
                J_{qp} & J_{qq}
            \end{bmatrix}
            \begin{bmatrix} v_p \\ v_q \end{bmatrix}
    \end{aligned}
\end{equation}
where $\tilde{\bf v}$ is the set of corrected voltages produced by applying the calibration solution, ${\bf J}$ to the measured voltages, ${\bf v}$.

\subsection{RTS}

The Real Time System is one of the pieces of software that can be used to generate calibration solutions for a VCS observation.

\subsection{Hyperdrive}

\section{Beamforming}

\section{Applications}

\section{Utilities}

\chapter{Implementation}

\section{Code glossary}

\subsection{\texttt{az}}
\begin{description}
    \item[Name:] \hyperlink{sec:coordslocalsky}{Geographic azimuth}\index{azimuth!geographic}
    \item[Units:] radians
    \item[Functions:] \hyperlink{cn:calc_beam_geom}{\texttt{calc\_beam\_geom}}
\end{description}

\subsection{\texttt{el}}
\begin{description}
    \item[Name:] \hyperlink{sec:coordslocalsky}{Elevation}\index{elevation}
    \item[Units:] radians
    \item[Functions:] \hyperlink{cn:calc_beam_geom}{\texttt{calc\_beam\_geom}}
\end{description}

\subsection{\texttt{za}}
\begin{description}
    \item[Name:] \hyperlink{sec:coordslocalsky}{Zenith angle}\index{zenith angle}
    \item[Units:] radians
    \item[Functions:] \hyperlink{cn:calc_beam_geom}{\texttt{calc\_beam\_geom}}
\end{description}

\section{Function reference}

\subsection{\texttt{calc\_beam\_geom}}
\label{fcn:calc_beam_geom}

\printindex

\end{document}
